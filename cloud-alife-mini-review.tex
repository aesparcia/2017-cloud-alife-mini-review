%%%%%%%%%%%%%%%%%%%%%%%%%%%%%%%%%%%%%%%%%%%%%%%%%%%%%%%%%%%%%%%%%%%%%%%%%%%%%%%%%%%%%%%%%%%%%%%%%%%%%%%%%%%%%%%%%%%%%%%%%%%%%%%%%%%%%%%%%%%%%%%%%%%%%%%%%%%
% This is just an example/guide for you to refer to when submitting manuscripts to Frontiers, it is not mandatory to use Frontiers .cls files nor frontiers.tex  %
% This will only generate the Manuscript, the final article will be typeset by Frontiers after acceptance.   
%                                              %
%                                                                                                                                                         %
% When submitting your files, remember to upload this *tex file, the pdf generated with it, the *bib file (if bibliography is not within the *tex) and all the figures.
%%%%%%%%%%%%%%%%%%%%%%%%%%%%%%%%%%%%%%%%%%%%%%%%%%%%%%%%%%%%%%%%%%%%%%%%%%%%%%%%%%%%%%%%%%%%%%%%%%%%%%%%%%%%%%%%%%%%%%%%%%%%%%%%%%%%%%%%%%%%%%%%%%%%%%%%%%%

%%% Version 3.3 Generated 2016/11/10 %%%
%%% You will need to have the following packages installed: datetime, fmtcount, etoolbox, fcprefix, which are normally inlcuded in WinEdt. %%%
%%% In http://www.ctan.org/ you can find the packages and how to install them, if necessary. %%%
%%%  NB logo1.jpg is required in the path in order to correctly compile front page header %%%

\documentclass[utf8]{frontiersSCNS} % for Science, Engineering and Humanities and Social Sciences articles
%\documentclass[utf8]{frontiersHLTH} % for Health articles
%\documentclass[utf8]{frontiersFPHY} % for Physics and Applied Mathematics and Statistics articles

%\setcitestyle{square} % for Physics and Applied Mathematics and Statistics articles
\usepackage{url,hyperref,lineno,microtype,subcaption}
\usepackage[onehalfspacing]{setspace}

\linenumbers


% Leave a blank line between paragraphs instead of using \\


\def\keyFont{\fontsize{8}{11}\helveticabold }
\def\firstAuthorLast{Sample {et~al.}} %use et al only if is more than 1 author
\def\Authors{Juan-Julián Merelo-Guervós\,$^{1,2,*}$, Co-Author\,$^{2}$ and Co-Author\,$^{1,2}$}
% Affiliations should be keyed to the author's name with superscript numbers and be listed as follows: Laboratory, Institute, Department, Organization, City, State abbreviation (USA, Canada, Australia), and Country (without detailed address information such as city zip codes or street names).
% If one of the authors has a change of address, list the new address below the correspondence details using a superscript symbol and use the same symbol to indicate the author in the author list.
\def\Address{$^{1}$Departamento de Arquitectura y Tecnología de
  Computadores, ETSIIT, Universidad de Granada, Spain \\
$^{2}$ CITIC, Universidad de Granada, Spain }
% The Corresponding Author should be marked with an asterisk
% Provide the exact contact address (this time including street name and city zip code) and email of the corresponding author
\def\corrAuthor{JJ Merelo, jmerelo@ugr.es}

\def\corrEmail{jmerelo@ugr.es}


\begin{document}
\onecolumn
\firstpage{1}

\title[Alife and cloud computing: a short review]{Alife in the clouds: a short review of applications of
  artificial life to cloud computing and back} 

\author[\firstAuthorLast ]{\Authors} %This field will be automatically populated
\address{} %This field will be automatically populated
\correspondance{} %This field will be automatically populated

\extraAuth{}% If there are more than 1 corresponding author, comment this line and uncomment the next one.
%\extraAuth{corresponding Author2 \\ Laboratory X2, Institute X2, Department X2, Organization X2, Street X2, City X2 , State XX2 (only USA, Canada and Australia), Zip Code2, X2 Country X2, email2@uni2.edu}


\maketitle


\begin{abstract}
  \section{}
Cloud computing is currently the prevailing mode of designing, creating and
deploying complex applications, and it has implied a paradigm shift in
all three areas; even if it taps and extends previous concepts such as
service oriented, concurrent and distributed computing, this shift has
to be eventually translated to the algorithmic and conceptual aspects
of sciences such as artificial life. In this short review we will make
a review of how the world of cloud computing has intersected the
artificial life field, and how it has been used as
inspiration for new models or implementation of new and powerful
algorithms; in the other direction, we will also examine how
artificial life concepts such as self-organization have been applied
to the field of cloud computing.  

\tiny
 \keyFont{ \section{Keywords:} Artificial life, cloud computing,
   distributed computing, complex systems. } %All article types: you may provide up to 8 keywords; at least 5 are mandatory.
\end{abstract}

\section{Introduction}

From its start less than ten years ago, cloud computing
\citep{armbrust2010view,qian2009cloud} has become in the last two
years the dominant computing platform. Even if initially it was a
metaphor applied to virtualized resources of any kind that could be
accessed in a pay-per-use basis, it has extended itself way beyond its
initial and simple translation of data center concepts to create
completely new software architectures and methodologies for
developing, testing and deploying large scale applications. Several
concepts account from the difference of cloud computing with respect
to other, {\em classical}, methodologies. 


\begin{itemize}

\item Infrastructure is fully automatized and described by software;
  there is no {\em hard} boundary between software and hardware, with
  hardware descriptions and applications that run on them
  developed at the same time.  

\item Applications are a loose collection of resources which interact
  asynchronously, are independent of each other, and in many cases
  have their own vendors or product owners. Some resources are
  ephemeral, appearing when they are needed and vanishing afterwards,
  some are permanent, but all of them behave {\em reactively}, being active when
  they receive information, and asynchronously, never waiting for
  response to return but relying instead on promises or other
  programming language constructs to deliver results to caller. 

\item  As such a loose collection, cloud-native applications might
  become {\em organic}  with parts of them changing continously, in a
  process called {\em continuous integration}. This complex internal
  structure and evolution make them, in fact, complex systems that
  have to be analyzed and designed with that in mind. 
\end{itemize}

On the other hand, artificial life \citep{wiki:alife} includes the
study of these kind of complex systems looking for similarities among
them and living systems, ecosystems or societies, in particular, what
is called {\em soft} artificial life looks at living systems to create
models and algorithms in engineering applications, from optimization
to conteng generation. The almost organic and decidedly complex nature
of large-scale cloud deployments and the fact that they constitute a
techno-social system \citep{vespignani2009predicting,JJ2016} hints at some relationship between
both scientific and engineering fields; however,
for the time being that has not been the case to a large extent, to the point
that the recent ECAL 2017 conference \citep{ecal17} only has a passing reference to
cloud systems.

That is the reason why we have decided to create this
mini-review of how cloud computing has been applied in the
alife/complex systems field, with an emphasis on those papers written
in the last few years, when the cloud ecosystem has experimented a big
diversity expansion, mainly due to the introduction of containers,
isolated applications that act like lightweight virtual machines. We
will do so in the next section.

\section{First steps towards cloud as complex systems}

One of the most straightforward ways of connecting {\em old} algorithms to
{\em new} technologies is simply to run those algorithms using a
straightforward translation of the old technologies. Cloud computing,
for instance, offers the possibility of instantiating and using computing nodes
at a cost that is a fraction of buying them, or even at no cost if you
land an academic grant or your problem falls within the boundaries of
the free tier cloud hosts offer. Using these nodes in the same way we
used computers in a data center or under our desktop, it is 
straightforward to port a parallel implementation of
whatever algorithm to a cloud-based one; this presents some challenges and
has in fact been addressed in several ones; for instance \citep{Medel2017}
studies how to create models of complex systems using cloud resources,
 \citep{merelo2011evostar} presents a system that uses free cloud
storage services as a device for interchanging individuals in an
evolutionary algorithm; Kopp \citep{Kopp2016} reviews technologies
that are, in general, available for simulation, focusing on defense
applications; however, simulation of adaptive behavior and agent-based
systems form the basis of artificial life and it could be applied to
it. Several other artificial life systems that use 
the cloud for storage are presented in a recent review
\citep{taylor2016webal}. 


There are other ways of translating {\em classical} frameworks and
algorithms, coming mainly from the fact that cloud computing is an
extension of service computing, that is, offering application
interfaces for a variety of services over the web. One of the possibilities that cloud computing offers
is the virtualization of computing resources and offering them {\em as a
service}. For instance, offering robot evolution as a service
\citep{du2017robot,chen2010robot} so that anyone can work with them without the
need to set up their own infrastructure. These new
implementations have their own scaling and scheduling issues which can
be a challenge, but they are mostly a straightforward shift
of $x$-as-a-framework-or-implementation to $x$-as-a-service. Besides,
as shown in the paper mentioned above, offering artificial life
frameworks as a service contributes to a
reduction of costs, which is one of the first-order results of working
in the cloud, but still it is only a short step away from
service-oriented architectures.  


The downside of using the cloud is the need of a permanent connection
with possibly a high bandwidth. However, there is 
an interesting complex-systems approach to this: so-called {\em edge
computing} \citep{satyanarayanan2017edge} moves cloud resources {\em close} to
the device or user consuming the service via technologies that allow them to access other
services using peer to peer technologies, or using mesh networks or
similar technologies to establish services that users of mainly mobile
devices can access. Since these {\em edge} nodes kind of {\em surround} the
user, they receive the denomination of {\em fog} computing too
\citep{luan2015fog}, which is studied mainly in the context of the Internet
of Things. The devices and computing nodes constituting this {\em fog}
are, effectively, a complex adaptive system
\citep{yan2010application,roca2018tackling}. However, the scale and
sheer 
number of devices used in fog computing exceed by orders of magnitude
the one in actual cloud computing system for the time being, so that
for the time being these concepts are still not being applied to
them. It is just, however, a matter of time and scale when cloud
systems will be considered self-organizing and exhibit emergent
behavior. We will refer to this in the next section.


\section{Complex cloud systems }

The possibilities of cloud architectures to exhibit some form of soft
artificial life were explored from early on, although mainly in blogs
and trade conferences \citep{burela09:Azure,hakimi}; this last one
mentions the cloud as a complex adaptive system, with most of the
features that these systems have such as adaptivity or
resiliency.
However, since the cloud systems are in principle
human-designed, that adaptivity and resiliency does not necesarily
exist. Complex adaptive cloud systems would make use of the underlying physical (or
virtual) characteristics of the cloud to implement complex
systems, although baseline cloud systems have already been studied and
modeled as CASs \citep{chen2013cloud}. 

Using the underlying characteristics of the cloud systems would
contribute to adaptivity and environment sensitiviy.
Most cloud systems architectures are based on queues or event systems, which
usually form the backbone of the whole cloud architecture and are used to deliver
information as well as activate different modules in a way that scales
well. This feature is leveraged in \citep{bottone2016implementing}, which
implements an ant-based system using the MQTT queue management
framework to deposit virtual pheromones, which are then {\em sniffed} by
agents in an stigmergic algorithm. Instead of simply implementing an
array in one of the virtual nodes or a database which can be read from
any client, this is a way that translates bioinspired algorithms to
their closest equivalent in a virtual environment, creating cloud
native complex adaptive systems. 

This can be taken further; every module in a cloud system is not
adaptive, performing whatever service it has been programmed to do via
its description. But lately, some cloud systems are being built with
self-* principles in mind; as services in the cloud acquire autonomy
in what is called {\em autonomic computing}, {\em adaptation} is not far away
and services become adaptive, being able to self-organize and the
whole system to self-heal, two agencies that make a cloud system
acquire complex adaptive behaviour, such as being able to reconfigure
connections, spawn new copies or eliminate them, all in an autonomic
way without needing to rely on a central authority, using peer to peer
protocols such as the {\em gossip} protocol \citep{LNCS44480129}, which has
actually been used for evolutionary algorithms \citep{laredo09cache} before
cloud computing was even created;  Briscoe and De Wilde
\citep{DBLP:journals/corr/abs-1101-5428,DBLP:journals/corr/abs-0712-4102}
  talk about {\em digital ecosystems} in the context of service based
  architectures, referring to distributed systems that evolve via
  agents mapped to nodes. This is a pioneering approach, which could
  form the basis of complex cloud systems. However, these approaches
  did not have a continuation when the whole plethora of cloud
  services started to become popular in the next years. 

\section{Conclusions}

As can be observed by the dearth of references for complex adaptive cloud
systems or cloud-native artificial life, as opposed to web-native
artificial life \citep{taylor2016webal}, we are still in the early
stages of its development, with the cloud being used mainly as a
resource for the straightforward porting of earlier systems. Some
steps have been taking in the realization of the complex adaptive
nature of systems and in the application of complex systems methods
such as stigmergy or gossip protocols to cloud native applications,
but it is still an early stage of development. The next few months or
years will undoubtedly bring new and unexpected developments to this
field that will provide benefits in both directions: insights in the
study of cloud as socio-technical systems and new ways of implementing
cloud-native artificial life forms.

\section{Acknowledgements*}

This work has been supported in part by the Spanish Ministry of Economía y Competitividad, projects TIN2014-56494-C4-3-P (UGR-EPHEMECH).


\bibliographystyle{frontiersinSCNS_ENG_HUMS} % for Science, Engineering and Humanities and Social Sciences articles, for Humanities and Social Sciences articles please include page numbers in the in-text citations
%\bibliographystyle{frontiersinHLTH&FPHY} % for Health, Physics and Mathematics articles
\bibliography{cloud-alife,geneura}


\end{document}
